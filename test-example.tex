\documentclass{article}
\usepackage{feynmp}
%\usepackage{pst-pdf}
\pagestyle{empty}

\DeclareGraphicsRule{*}{mps}{*}{}

\begin{document}
\unitlength = 1mm	% Defines unit length for Feynman Diagrams

%%%%%%%%%%%%%%%%%   FIGURE   %%%%%%%%%%%%%%%%%
\begin{figure}[h!tbp]
\centering
\bigskip
\begin{fmffile}{example}
\begin{fmfgraph*}(40,25)
%%%% VERTICIES %%%%
\fmfleft{i1,i2}
\fmfright{o1,o2}
%%%% PARTICLES %%%%%%
\fmflabel{$p_2^i$}{i1}
\fmflabel{$p_1^i$}{i2}
\fmflabel{$p_2^f$}{o1}
\fmflabel{$p_1^f$}{o2}
%%%%INTERACTIONS%%%%%
\fmf{plain}{i1,v1}
\fmf{plain}{i2,v2}
%%%
\fmf{plain,label=$q$,left=0.4,tension=0.2}{v1,v2}
\fmf{plain,label=$k$,right=0.4,tension=0.2}{v1,v2}
%%%
\fmf{plain}{v1,o1}
\fmf{plain}{v2,o2}
%%%
\fmfdot{v1,v2}
%%%%%%%%%%%%%%
\end{fmfgraph*}
\end{fmffile}
\bigskip
%\caption{Put the caption in your Figure instead}
%\label{fig:example}
\end{figure}
%%%%%%%%%%%%%%%%%%%%%%%%%%%%%%%%%%%%%%%%

\end{document}